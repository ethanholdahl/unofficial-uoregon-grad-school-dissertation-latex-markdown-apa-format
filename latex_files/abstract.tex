\abstract{
This dissertation examines game theory and evolutionary dynamics, exploring strategic decision-making, social norm emergence, and inter-group conflicts.

Chapter 2 focuses on stepping stones, recurrent classes that facilitate equilibrium transitions. An experiment tests their effectiveness in promoting the transition to a Pareto efficient equilibrium. Results show groups with stepping stones consistently achieve the high-payoff equilibrium, contrasting occasional failures in groups without them. Information about other players' payoffs is crucial, with complete information outperforming incomplete information. However, the effect diminishes with stepping stones, emphasizing their low-cost transitions. Players' decision-making behavior and factors influencing deviations are also examined.

Chapter 3 explores the role of incomplete sampling in determining convergence to conventions in adaptive play. The chapter demonstrates that even minimal incomplete sampling is sufficient for convergence to occur in the $2\times 2$ coordination game. The analysis also reveals that incomplete sampling criteria are often unnecessary, expanding the boundaries of adaptive play theory. The implications of incomplete sampling on the perturbed adaptive process are examined, identifying a robust resistance function that persists under different degrees of sampling.

In Chapter 4, the effects of signaling in inter-group conflicts are investigated. The competitive advantage of costly signaling within groups is examined, and a model is developed to explore the dynamics of inter-group conflicts. The findings suggest that shorter periods of isolation and more efficient weapons favor the rise of signaling norms in societies.

Overall, this dissertation provides valuable insights into game theory, evolutionary dynamics, and their implications for strategic decision-making, social norms, and inter-group conflicts. The findings contribute to interdisciplinary fields such as economics, sociology, and political science, offering a foundation for further research in these areas.
	
	This dissertation includes both previously published co-authored material and unpublished co-authored material.
	
}

